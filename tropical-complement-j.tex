\documentclass[a4paper,dvipdfmx,reqno,12pt]{amsart}
%\documentclass[a4paper,reqno,12pt]{amsart}%for arxiv
\synctex=1
%
%%%% packages
\usepackage[utf8]{inputenc}

\usepackage{mathtools,thmtools}
%\usepackage{pgf,pgfplots}
%\usepackage{tikz}
%\pgfplotsset{compat=1.15}
%\usetikzlibrary{arrows}
\usepackage{tikz-cd}%
%\usetikzlibrary{cd}
%\usetikzlibrary{calc}

\usepackage{graphicx,color}%for images
\usepackage{bm}%fonts
\usepackage{amsmath,amsthm,amstext,amsfonts,amsbsy}%
\usepackage{amssymb}
\usepackage{latexsym}%
%\usepackage{algpseudocode,algorithm}%
\usepackage{todonotes}%comments
\usepackage[margin=3cm]{geometry}
\usepackage{layout}
\usepackage[T1]{fontenc}%font encoding
\usepackage{physics}
\usepackage{braket}%after physics

% \usepackage{imakeidx}%before hyperref for pagebackref
\usepackage[pagebackref]{hyperref}
\usepackage[capitalize]{cleveref}
\hypersetup{
     colorlinks = true,
     citecolor  = blue,
     linkcolor  = blue, 
     urlcolor   = blue, 
}
%\usepackage{pxjahyper}%for hyperref in Japanese
\usepackage{bookmark}

\usepackage{fancyhdr}
\pagestyle{plain}
\setlength{\footskip}{30pt}

%%%%


%%%% theoremstyle

\theoremstyle{definition}
\newtheorem{theorem}{Theorem}[section]
\newtheorem*{theorem*}{Theorem}
\newtheorem{definition}[theorem]{Definition}
\newtheorem{definition*}{Definition}
\newtheorem{example}[theorem]{Example}
\newtheorem*{example*}{Example}
\newtheorem{proposition}[theorem]{Proposition}
\newtheorem*{proposition*}{Proposition}
\newtheorem{note}[theorem]{Note}
\newtheorem*{note*}{Note}
\newtheorem{notice}[theorem]{Notice}
\newtheorem*{notice*}{Notice}
\newtheorem{lemma}[theorem]{Lemma}
\newtheorem*{lemma*}{Lemma}
\newtheorem{fact}[theorem]{Fact}
\newtheorem*{fact*}{Fact}
\newtheorem{question}[theorem]{Question}
\newtheorem*{question*}{Question}
\newtheorem{conjecture}[theorem]{Conjecture}
\newtheorem*{conjecture*}{Conjecture}
\newtheorem{notation}[theorem]{Notation}
\newtheorem*{notation*}{Notation}
\newtheorem{corollary}[theorem]{Corollary}
\newtheorem*{corollary*}{Corollary}
\newtheorem{remark}[theorem]{Remark}
\newtheorem*{remark*}{Remark}
\newtheorem{condition}[theorem]{Condition}
\newtheorem*{condition*}{Condition}
\newtheorem{convention}[theorem]{Convention}
\newtheorem*{convention*}{Convention}
\newtheorem{observation}[theorem]{Observation}
\newtheorem*{observation*}{Observation}
%%%% newcommand

%%%logic symbol
\newcommand{\deq}{\coloneqq}

\newcommand{\dbraket}[1]{\hspace{-1.5pt}\braket{\hspace{-2.2pt}\braket{#1}\hspace{-2.2pt}}}

\newcommand{\textcmd}[1]{\texttt{\symbol{"5C}#1}}

%%special sets
\newcommand{\emp}{\varnothing}%emptyset
\newcommand{\C}{\mathbb{C}}%complex number
\newcommand{\Ha}{\mathbb{H}}%quaternion
\newcommand{\F}{\mathbb{F}}%field
\newcommand{\R}{\mathbb{R}}%real number
\newcommand{\Q}{\mathbb{Q}}%rational number
\newcommand{\Z}{\mathbb{Z}}%integer
\newcommand{\N}{\mathbb{N}_{0}}%natural number
\newcommand{\Pj}{\mathbb{P}}%bold p
\newcommand{\vep}{\varepsilon}%varepsilon

%%%%

\newcommand{\mb}[1]{\mathbb{#1}}%blackboard bold (for math mode)
\newcommand{\mcal}[1]{\mathcal{#1}}%

\newcommand{\opn}[1]{\operatorname{#1}}
\newcommand{\catn}[1]{\mathbf{#1}}

\newcommand{\abk}[1]{\langle {#1} \rangle}%angle bracket 
\newcommand{\Abk}[1]{\left \langle {#1} \right \rangle}%angle bracket (auto sizing)
\newcommand{\dabk}[1]{\langle\! \langle {#1}\rangle \! \rangle}%double angle bracket
\newcommand{\Dabk}[1]{\left \langle \! \left \langle {#1} \right \rangle \! \right \rangle}%double angle bracket
\newcommand{\Sbk}[1]{\left[ {#1} ]\right }% square bracket [] (auto sizing)
\newcommand{\Cbk}[1]{\left \{ {#1}\right \}}% curly bracket {} (auto sizing)
\newcommand{\dcbk}[1]{\{\!\!\{ {#1}\}\!\!\}} % double curly bracket {{}} 
\newcommand{\Dcbk}[1]{\left \{\!\! \left \{ {#1} \right\} \!\!\right \}} % double curly bracket {{}} (auto sizing)
\newcommand{\Paren}[1]{\left ( {#1} \right )}%parenthesis () (auto sizing)
\newcommand{\dparen}[1]{(\!({#1})\!)}%double parenthesis
\newcommand{\xto}[1]{\xrightarrow{#1}}
\newcommand{\xgets}[1]{\xleftarrow{#1}}
\newcommand{\hookto}{\hookrightarrow}


%%%% 

% %%%% mathabx.sty (font) 
% \DeclareFontFamily{U}{matha}{\hyphenchar\font45}
% \DeclareFontShape{U}{matha}{m}{n}{
%       <5> <6> <7> <8> <9> <10> gen * matha
%       <10.95> matha10 <12> <14.4> <17.28> <20.74> <24.88> matha12
%       }{}
% \DeclareSymbolFont{matha}{U}{matha}{m}{n}

% \DeclareFontFamily{U}{mathb}{\hyphenchar\font45}
% \DeclareFontShape{U}{mathb}{m}{n}{
%       <5> <6> <7> <8> <9> <10> gen * mathb
%       <10.95> mathb10 <12> <14.4> <17.28> <20.74> <24.88> mathb12
%       }{}
% \DeclareSymbolFont{mathb}{U}{mathb}{m}{n}

% \DeclareFontFamily{U}{mathx}{\hyphenchar\font45}
% \DeclareFontShape{U}{mathx}{m}{n}{
%       <5> <6> <7> <8> <9> <10>
%       <10.95> <12> <14.4> <17.28> <20.74> <24.88>
%       mathx10
%       }{}
% \DeclareSymbolFont{mathx}{U}{mathx}{m}{n}

% %DeclareMathSymbol (from mathabx.sty)
% \DeclareMathSymbol{\bigboxslash}{\mathop}{mathx}{"FE}
% \DeclareMathSymbol{\bigboxtimes}{\mathop}{mathx}{"D2}
% %%%%

% %%%% MnSymbol.sty (font)
% \DeclareFontFamily{U}{MnSymbolC}{}
% \DeclareFontShape{U}{MnSymbolC}{m}{n}{
%   <-6> MnSymbolC5
%   <6-7> MnSymbolC6
%   <7-8> MnSymbolC7
%   <8-9> MnSymbolC8
%   <9-10> MnSymbolC9
%   <10-12> MnSymbolC10
%   <12-> MnSymbolC12}{}
% \DeclareFontShape{U}{MnSymbolC}{b}{n}{
%   <-6> MnSymbolA-Bold5
%   <6-7> MnSymbolC-Bold6
%   <7-8> MnSymbolC-Bold7
%   <8-9> MnSymbolC-Bold8
%   <9-10> MnSymbolC-Bold9
%   <10-12> MnSymbolC-Bold10
%   <12-> MnSymbolC-Bold12}{}

% \DeclareSymbolFont{MnSyC}{U}{MnSymbolC}{m}{n}

% %%%% DeclareMathSymbol (from MnSymbol.sty)

% \DeclareMathSymbol{\tplus}{\mathbin}{MnSyC}{43}
% \DeclareMathSymbol{\aplus}{\mathbin}{MnSyC}{190}

%%%% renewcommand




%%%% footnote

\newcommand{\myfootnote}[1]{\hspace{-5pt}\footnote{#1}}

\newcommand{\TB}{\mcal{T}_{B}}
\newcommand{\TBZ}{\mcal{T}_{\Z,B}}
\newcommand{\AffS}{{\mathop{\mcal{A}\!f\!\!f\!}\nolimits}}
\newcommand{\simto}{ 
\mathrel{\raisebox{0.13em}{${\sim}$}}
\kern -0.75em \mathrel{\raisebox{-0.11em}{${\scriptstyle \to}$}}  
}
%%%%  

%%%% 

\renewcommand*{\backrefalt}[4]{%
\ifcase #1 %
\or        [Cited on p.#2.]%
\else      [Cited on pp.#2.]%
\fi}

\DeclareMathOperator{\Pic}{Pic}
\DeclareMathOperator{\CDiv}{Div^{\infty}}
\DeclareMathOperator{\fuk}{fuk}
\DeclareMathOperator{\coh}{coh}
\DeclareMathOperator{\reg}{gen}
\DeclareMathOperator{\tform}{\Omega}
\DeclareMathOperator{\mform}{\mathcal{F}}

\crefname{equation}{}{}
\crefname{conjecture}{Conjecture}{Conjectures}

\usepackage{mathrsfs}
\usepackage{upgreek}
\numberwithin{equation}{section}
\title{トロピカル多様体の超曲面の補空間と Riemann--Roch 数}
\author[Y. Tsutsui]{Yuki Tsutsui}
\address{Graduate School of Mathematical Sciences,
The University of Tokyo, 3-8-1 Komaba, Meguro-Ku,
Tokyo, 153-8914, Japan}
\email{tyuki@ms.u-tokyo.ac.jp}
%\date{\today}

\begin{document}

\maketitle

\section{イントロダクション}

\cite{demedrano2023chern}の理論によって
トロピカル多様体の Todd 類 が定義可能となった。
特に、
コンパクトトロピカル多様体$X$上の因子
$D$ の
Riemann--Roch 数 $\opn{RR}(X;D)$ が
次のように定義できるようになった。

\begin{align}
     \opn{RR}(X;D)\deq \int_{X}\opn{ch}(\mathcal{L}(D))\opn{td}(X).
\end{align}
ここで、$\mathcal{L}(D)$とは、$D$が定める
トロピカル直線束であり、
$\opn{ch}(\mathcal{L}(D))\deq 
\sum_{i=0}^{\infty}\frac{c_1(\mathcal{L}(D))^{i}}{i!}$
は、$\mathcal{L}(D)$ の Chern 指標であり、
$\int_X$ はトレース写像である。
古典的には、この Riemann--Roch 数は $\opn{RR}(X;D)$
は$D$の定義する可逆層の Euler 数と対応する。
しかし、トロピカル幾何では、
可逆層とくに構造層は Abel 群のなす層では
ないため、可逆層の層コホモロジーの Euler 数と
Riemann--Roch 数を比較することができない。
いづれにせよ、
$\opn{RR}(X;D)$ がトロピカル幾何的にどういう意味を持った
量であるのかということの理解は、
トロピカル幾何における
Riemann--Roch の定理の定式化および解釈を与えるうえで
重要な話題であるといえる。

少なくとも$\opn{RR}(X;0)$の幾何学的な意味は、
専門家の間で共通の認識が存在し、
\cite[Conjecture 6.13]{demedrano2023chern}
では、$\opn{RR}(X;0)$は、$X$
の位相的 Euler 数 $\chi(X)$ と一致していると
予想されており、
$X$ が Delzant face 構造を持つトロピカル曲面の場合
において\cite[Theorem 6.3]{demedrano2023chern}にて
証明された。なお Delzant face 構造を持つという
条件は、十分に広い範囲のトロピカル曲面で成立している
\cite[Corollary 6.11]{demedrano2023chern}。

なお、トロピカル多様体の位相的 Euler 数
$\chi(X)$が代数多様体上の構造層の
コホモロジーの Euler 数のトロピカル幾何的な類似であるべき
であることは、古くから期待されており、
\cite[Corollary 2]{MR3961331}
などで実際に対応していることが示されている。
本論文では $\opn{RR}(X;D)$ のトロピカル幾何学的な
意味について考察することが主目的である。

--------イントロには要らないかも(はじまり)

筆者は\cite{tsutsui2023graded}にて、
ミラー対称性と超局所層理論のアイデアから
コンパクトトロピカル多様体上の
直線束を表現する$C^{\infty}$因子に対して
次数付き加群を定義し、それが Riemann--Roch 数と
一致するという予想\cite[Conjecture 1.2]{tsutsui2023graded}を
提案し、トロピカル多様体がトロピカル曲線や Hesse 計量を
許容する整アフィン多様体に対して成立することを示した。
$C^{\infty}$ 因子のコホモロジカル
局所 Morse データはSYZ変換のトロピカル類似によって
古典的な複素直線束の Riemann--Roch 数
と関連付けることができると期待している。
さらに、理想的には$C^{\infty}$-因子に対して
Floer コホモロジーが定義されると期待している。
本文書では、前述の組み合わせ版を考察する。
---------イントロには要らないかも(おわり)


特に本文書では、トロピカル曲面の場合が中心的な
話題であるが、これらの内容は高次元に一般化できると
期待している。

まず、代数幾何的な基本事項を整理する。
$X$を非特異射影多様体とし、$D$をその非特異因子とし、
$\iota\colon D\to X$を自然な閉埋め込みとする。
このとき、次の層短完全列が存在する。
\begin{align}
     0\to \mathcal{O}_X(-D)\to 
\mathcal{O}_X\to \iota_*\mathcal{O}_D\to 0
\end{align}
これより、次の等式が成り立つ。
\begin{align}
\chi(X;\mathcal{O}_X(-D))=\chi(X;\mathcal{O}_X)
-\chi(D;\mathcal{O}_D)
\end{align}

次の予想は、以上の事実を踏まえるとごく自然であり,
広く信じられていると思われる。

\begin{conjecture}[{Folklore?}]
\label{conjecture-rr-c-euler}
$X$ をコンパクトトロピカル多様体とし、$D$を
付随するサイクル $\opn{div}_X(D)$ が
余次元 $1$
のコンパクト
トロピカル部分多様体(tropical submanifold)
となる Cartier 因子とする
\cite[Definition 2.14]{demedrano2023chern}。
このとき、次の等式が成り立つ。
\begin{align}
\opn{RR}(X;-D)=
\chi_c (X\setminus D) (=\chi_c (X)-\chi_c(D))
\end{align}
ここで$\chi_c(X)$とは、$X$のコンパクト台のコホモロジーの
Euler 数であり、
$\chi_c(D)$ は、$D$ が定義する
$X$上のトロピカル $1$-サイクル $\opn{div}_X(D)$
の台 $|D|$ のコホモロジー
$\chi_c(X\setminus D)$は、
補空間 $X\setminus |D|$ の コンパクト台のコホモロジーの
Euler 数である。
\end{conjecture}

\begin{example}
$\dim X=1$ のときは、
\cref{conjecture-rr-c-euler}は自明である。

次に、$\dim X=2$とすると
トロピカル曲線の随伴公式
(\cite[Theorem 6]{shaw2015tropical}と
\cite[Theorem 5.2]{demedrano2023chern}を参照)より、
\begin{align}
\opn{RR}(X;-D)=\frac{(-D-K_X).(-D)}{2}+\opn{RR}(X;0)
=\chi(D)+\opn{RR}(X;0).
\end{align}

よって、$X$ が Delzant face 構造を持つとき、
\cref{conjecture-rr-c-euler} が成立する。
なお、ここではサイクルの交点数と付随する Chern 類の
交点数が一致することを用いている
(\cref{proposition-cycle-chern})。
\end{example}
本論文の主役は、$\opn{RR}(X;-D)$の双対である
$\opn{RR}(X;D)$の幾何学的意味を与える次の予想である。
(こちらも、\cref{conjecture-rr-c-euler}と
同様に他の研究者も予想していると思われるが、より非自明である。)
\begin{conjecture}
\label{conjecture-rr-euler}
$X$ をコンパクトトロピカル多様体とし、$D$を余次元$1$
のコンパクト
トロピカル部分多様体(tropical submanifold)とする
\cite[Definition 2.14]{demedrano2023chern}。
$D$が$X$上の中庸な配置
(\cref{definition-permissible-position})にあるとき、次が成り立つ。
\begin{align}
\opn{RR}(X;D)=\chi(X\setminus D).
\end{align}
\end{conjecture}

中庸な配置の条件は、十分に一般的なトロピカル部分多様体に対して
成立する条件である一方で、満たさないものも簡単に作れる条件である。
(\cref{example-permissible-point}などを参照。)

\begin{example}
\label{example-permissible-point}
$C$をコンパクトトロピカル曲線とする。点 $p\in C$ が
中庸な(permissible)位置にあることは、
$p\in C_{\opn{reg}}$ であることと同値である。
当然、有限個の点集合の場合に一般化できる。
簡単な計算から、中庸な位置にある部分トロピカル多様体
$D$に対して、
\begin{align}
\opn{RR}(C;D)=\sharp (\opn{supp}(D))+ \chi_{\opn{top}}(C)
=\chi_{\opn{top}}(C\setminus D).
\end{align}
よって、$\dim X=1$のとき、
\cref{conjecture-rr-euler}
は正しい。
\end{example}

\cref{conjecture-rr-euler}
が成り立つと期待される根拠となるその他の例は、
\cref{example-TPn} や
\cref{remark-iass} で説明する。

本論文の主定理の一つは次のようになる。
証明は難しくない。
\begin{theorem}[{Main theorem}]
\label{theorem-rr-euler-surface}
$X$ をコンパクトトロピカル曲面とし、
$D$を中庸な位置にある
余次元 $1$ のコンパクトトロピカル部分多様体
に対して次が成り立つ。
\begin{align}
\chi(X\setminus D)=\frac{(D-K_X). D}{2}+
\chi(X).
\end{align}
\end{theorem}
\cite[Theorem 6.3]{demedrano2023chern}
とを合わせると次のようになる。
\begin{corollary}
$X$が Delzant face 構造をもつコンパクトトロピカル
曲面のとき、 \cref{conjecture-rr-euler}は正しい。
\end{corollary}

もう少しだけ踏み込んだ話をする。
トロピカル多様体$X$の
余次元$1$のトロピカル部分多様体$D$
$\opn{RR}(X;D)$の幾何学的な意味を与えると、
次のようなもう少し一般的な因子に対しても
幾何学的な意味を与えることができると期待できる。
以下よりそれを説明する。

$X$を非特異射影多様体とし、$D$を非特異因子とし、
$\iota\colon D\to X$ を埋め込み写像とする。
任意の因子 $D'$に対して次の短完全列が得られる。
\begin{align}
0 \to \mathcal{O}_X(D'-D)\to \mathcal{O}_X(D')
\to \mathcal{O}_X(D')
\otimes_{\mathcal{O}_X} \iota_*\mathcal{O}_D \to 0 
\end{align}

一方で局所自由層の射影公式より、
\begin{align}
\chi(X;\mathcal{O}_X(D')\otimes_{\mathcal{O}_X} \iota_*\mathcal{O}_D)
=\chi(X;\iota_*(\iota^{*}\mathcal{O}_X(D')\otimes_{\mathcal{O}_D} \mathcal{O}_D))
=\chi(D;\iota^{*}\mathcal{O}_X(D'))
\end{align}
となり、$D'$が非特異因子で$D\cap D'$もまた$D$の非特異因子
になるならば、有効因子の引き戻しの性質として次のように
書き直せる。
\begin{align}
\chi(D;\iota^{*}\mathcal{O}_X(D'))=\chi(D;\mathcal{O}_D(D'\cap D))
\end{align}

上述の条件を満たす $D,D'$ への条件は厳しいように見えるが
Bertiniの定理の応用から射影多様体の任意の因子$D_0$に対して
$D_0=D'-D$かつ
$D,D',D\cap D'$ が非特異部分多様体になるように$D,D'$
を選ぶことができる。
そして、以上の仮定の下では次の等式が成り立つ。
\begin{align}
\chi(X;\mathcal{O}_X(D'-D))=
\chi(X;\mathcal{O}_X(D'))-
\chi(D;\mathcal{O}_D(D'\cap D))
\end{align}

当然ではあるが、Grothendieck--Riemann--Rochの定理からも
同様の公式が導出される。これと補集合に関する前述の予想を融合させると
次のことも期待できる。
本文書ではこの予想についても、
Noetherの公式を満たすトロピカル曲面の場合において示す。
\begin{conjecture}
\label{conjecture-rr-bertini}
$X$をコンパクトトロピカル多様体とし, 
$D,D'$を余次元$1$のトロピカル部分多様体または
空集合とする。
$D$ と $D'$ は次の条件を満たしているとする。
\begin{enumerate}
\item $D'$が$X$上中庸な位置にある。
\item $D$ と $D'$ は横断的かつ局所マトロイド的に
交わる。
\item $D\cap D'$ が $D$ 上中庸な位置にある。
\end{enumerate}

$D'$が$X$上中庸な位置にあるとし、
$D\cap D'$が$D$の(重みづけも含めて)
余次元$1$のトロピカル部分多様体または
空集合であり、$D\cap D'$ が$D$上中庸な位置にあると仮定する。
このとき、次の等式が成り立つ。
\begin{align}
\opn{RR}(X;D'-D)=\chi (X\setminus D')-
\chi(D\setminus (D'\cap D))
\end{align}

\end{conjecture}

\begin{remark}
$D,D'$が空集合のとき、
\cref{conjecture-rr-bertini}は、
\cite[Conjecture 6.13]{demedrano2023chern}
と同等である。また、$D'$が空集合で$D$が空ではないとき、
\cref{conjecture-rr-bertini}は、
\cref{conjecture-rr-c-euler}と同等であり、
$D$が空集合で$D'$が空ではないとき、
\cref{conjecture-rr-bertini}は、
\cref{conjecture-rr-euler}と同等である。
よって\cref{conjecture-rr-bertini}は上述の
三つの予想を一般化したものとみなすことができる。
\end{remark}

\begin{conjecture}
$X$をトロピカル射影空間の部分トロピカル多様体とする。
このとき、任意の因子$D_0$に対して
\cref{conjecture-rr-bertini}において課した
条件をみたし、$D_0=D'-D$となる
余次元$1$の部分トロピカル多様体の組
$D,D'$が存在する。
\end{conjecture}



なお後述するが\cref{conjecture-rr-bertini}は、
\cref{conjecture-rr-euler}と Todd 類に関する
満たすべき性質との組み合わせから帰結することが
できる。

右辺の量は随分具体的な量であるが、実は
$D$も$X$上の中庸な位置にあるとき、
$C^{\infty}$因子の局所Morseデータと
関連付けることが可能である。
局所 Morse データ(の一般化論)を用いることで
$\chi_c(X\setminus D)$ と
$\chi (X\setminus D)$
は統一的に管理することができる。
論文の最後にこのことを曲面の場合に説明する。

\subsection{本論文のアウトライン}



\section{証明の準備}

ここでは、本論文で必要な
トロピカル多様体の
前提知識を
\cite{mikhalkin2018tropical,MR4637248,demedrano2023chern}
を参考に整理する。
(必要に応じて、別の文献も引用する。)
本論文の後半を記述する上での事情から、
\cite{MR4637248}による
層理論的な定式化を主に採用することとする。
特に、本論文でのトロピカル多様体の定義は
\cite[Definition 6.1]{gross2019sheaftheoretic}
を採用し、finite type に関する条件
(e.g. \cite[Definition 7.1.14]{mikhalkin2018tropical} や
\cite[Definition 2.3 (4)]{demedrano2023chern})
は課さないこととする。トロピカル多様体がコンパクト
なら finite type に関する条件は気にする必要はない。

\begin{notation}
本文書での
有理多面空間 $(X,\mathcal{O}_X^{\times})$ 
\cite[Definition 2.2]{MR4246795} の
次元$\dim X$とは、$X$の局所コンパクト Hausdorff 空間
の次元\cite[Chapter III. Definition 9.4]{MR842190}
のことであり、点 $x\in X$ に対して
$\dim_x X$ を $X$ の $x$ における局所次元と呼ぶ
\cite[Chapter III. Definition 9.10]{MR842190}。
有理多面空間
$(X,\mathcal{O}_X^{\times})$
が、純次元的(pure dimensional)
とは、任意の点 $x\in X$ に対して
$\dim X=\dim_x X$ であることをいう。
これらの定義は、
\cite[Definition 7.1.1]{mikhalkin2018tropical}
と整合的である。
\end{notation}

\begin{remark}
有理多面空間はパラコンパクトかつ局所可縮である。
特に、定数層のコホモロジーと
特異コホモロジーが同型である。
有理多面空間が
boundaryless もしくは
トロピカル多様体であるときは自明である。
一般の有理多面空間が局所可縮であることは、
$\mathbb{T}^{n}$ 内の凸多面体 $P$ が
局所可縮であることと、
$\mathbb{R}^{n}$ 内の コンパクト集合 $K$ が Euclid 
neighborhood retract であることと
$K$が局所可縮であることが同値であること、
Euclid neighborhood retract な距離空間は
absolute neighborhood retract であること、
そして Borsuk's pasting theorem から示される。
(\href{https://eudml.org/doc/212574}{EUDML | Über eine Klasse von lokal zusammenhängenden Räumen}
の p.226 で初めて示されたらしい。
)

もしくは、
$\mathbb{T}^{n}$ 内の有理多面集合 $P$ の指数写像
$\opn{exp}\colon \mathbb{T}^{n}\to 
\mathbb{R}^{n}_{\geq 0}$ の像が $\mathbb{R}^{n}$
内の subanalytic set であることと、
$\mathbb{R}^{n}$ 内の局所閉な
subanalytic set に三角形分割が存在すること
\cite{MR760983} からも
示すことができる。
なお、subamalytic set の三角形分割の
存在定理は、$o$-極小構造の理論によってより一般化されている。
(詳しくは、\cite[Theorem II]{MR1463945} などを見よ。)
\end{remark}

\cite{demedrano2023chern}に倣い、
有理多面空間 $(X,\mathcal{O}_X^{\times})$
全体を被覆するチャートの集まり
$\{(U_i,\psi_i)\}_{i\in I}$を
$(X,\mathcal{O}_X^{\times})$のアトラスと呼ぶこととする。
有理多面空間$(X,\mathcal{O}_X^{\times})$ が
\cite[Definition 6.1]{gross2019sheaftheoretic}
の意味でトロピカル多様体であるとき、各チャート
$\psi_i\colon U_i\to V_i(\subset \underline{\mathbb{R}}^{n_i})$
の$V_i$が$L_M\times \underline{\mathbb{R}}^{r_i}$の開集合
となるようなアトラス$\{(U_i,\psi_i)\}_{i\in I}$を
取ることができる。

\cite[Definition 2.14]{demedrano2023chern}
に倣い、本文における部分有理多面空間
(rational polyhedral subspace)
の定義を以下にしておく。


\begin{definition}

$(X,\mathcal{O}_X^{\times})$ を有理多面空間とし,
$Y$を部分空間とし, 
$\mathcal{O}^{\times}_{Y}\deq 
\opn{Im}(i^{-1}\mathcal{O}_X^{\times}\to \mathcal{C}_{Y}^{0})$
とする。
$(Y,\mathcal{O}_Y^{\times})$ が有理多面空間となり、
$Y$ のアトラスが、
$(X,\mathcal{O}^{\times}_X)$
のアトラスの $Y$ への制限として得られるとき、
$(Y,\mathcal{O}_Y^{\times})$ を
$X$ の部分有理多面空間と呼ぶ。
\end{definition}

$X$が第二可算 Hausdorff なので
$Y$もまた第二可算 Hausdorff である。
よって、整合的なチャートが存在することさえ確認すればよい。

$(X,\mathcal{O}_X^{\times})$ を有理多面空間
$(Y,\mathcal{O}_Y^{\times})$ を $X$ の
部分有理多面空間とする。
$\iota\colon Y \to X$ を包含写像としたとき、
各点 $x\in Y$ ごとに局所錐の包含射
$\iota_{*,x}\colon \opn{LC}_x Y\to \opn{LC}_x X$
が自然と誘導される。
さらに、$(X,\mathcal{O}_X^{\times})$ と
$(Y,\mathcal{O}_Y^{\times})$ が
regular at infinity 
(\cite[\textsection 6.1]{MR4637248} や
\cite[Definition 1.2]{MR3330789} を見よ) 
であるとき、各点 $x$ の近傍の射はある射
$\iota_{x}\colon \opn{LC}_x Y\times 
\underline{\mathbb{R}}^{r_{\alpha}}
\to \opn{LC}_x X\times \underline{\mathbb{R}}^{r_{\alpha'}}$
の制限として得られる。

特に、十分に小さい近傍において$\opn{Aff}_X(U)$は、
$T_x X\times \underline{\mathbb{R}}^{r_{\alpha}}$
上の大域的な$\mathbb{Z}$-アフィン関数と全単射である。
(\cite[Example 2.1]{MR4637248} でも説明されているように、
大域的ではない局所$\mathbb{Z}$-アフィン関数も存在する。)
特に、$(X,\mathcal{O}_X^{\times})$が regular at infinity
ならば、座標変換は
extended integer affine map \cite[Definition 2.2]{demedrano2023chern}
となることがこれよりわかる。よって
$(X,\mathcal{O}_X^{\times})$ 
が\cite[Definition 6.1]{gross2019sheaftheoretic}の意味での
コンパクトトロピカル多様体であるとき、
\cite[Definition 2.3]{demedrano2023chern}
の意味でのトロピカル多様体の構造を与えるアトラスを持つ。


\begin{definition}[{\cite[Definition 2.14]{demedrano2023chern}}]
$X$をトロピカル多様体とする。
あるトロピカルサイクルの台 $Y$ が、
トロピカル部分多様体とは、
$Y$が部分有理多面空間として
トロピカル多様体かつ、
任意の点$x\in Y$の
局所錐の包含射
$\iota_{*,x}\colon \opn{LC}_x Y\to 
\opn{LC}_x X$
がマトロイド商から誘導される
Bergman 扇の包含射と同値
であるものをいう。
\end{definition}


トロピカルサイクルの台は、閉部分集合なので
部分トロピカル多様体は常に閉集合である。

\begin{remark}
\cite[Example 2.15]{demedrano2023chern}でも強調されているように、
トロピカル多様体の閉部分多面空間がトロピカル多様体だとしても
トロピカル部分多様体であるとは限らない。
例えば、
\cite[Example 2.21]{shaw2015tropical}
における $C$ は、 valent が $3$ のトロピカル曲線
であるが、$P$ の部分トロピカル多様体ではない。
実際にこの埋め込みは、台集合を同じとするマトロイドに対する
包含関係$L_M\subset L_N$に由来しない。
その他の例については、
\cite{MR2594592} や \cite{MR3339531}
などを参照せよ。


\end{remark}

\begin{remark}
$Y$が$X$のトロピカル部分多様体のとき、
トロピカルサイクルの台なので、$Y$は
$X$の閉部分集合である。
逆に、$Y$が、$X$
の有理多面閉部分空間かつトロピカル多様体ならば、
閉埋め込み写像 $\iota\colon Y\to X$ 
の押し出し
\cite[\textsection 3.2]{demedrano2023chern}
によって、$\iota_*1_{Y_{\mathrm{reg}}}\in Z_{\bullet}(X)$
となる。
\end{remark}

$X$をトロピカル多様体のとし、
$\opn{sed}_X\colon 
X\to \mathbb{Z}$ を
$X$ 上の sedentarity 関数
\cite[Definition 7.2.6]{mikhalkin2018tropical}
とする。
(\cite[Definition 2.4]{demedrano2023chern}
もみよ。)
定義より、任意の点 $x\in X$ に対して
次の等式が成り立つ。
\begin{align}
\opn{sed}_X(x)+\dim \opn{LC}_x X=\dim_x X.
\end{align}
また $\opn{sed}_X$ は、
上半連続関数である。
特に次の集合たちは、
$X$の閉部分集合である
(上半連続関数の定義は例えば\cite[p.287]{MR463157})。
\begin{align}
X^{[\geq k]}\deq \{p\in X\mid \opn{sed}_X(p)\geq k\},
\quad 
X_{\infty}\deq X^{[\geq 1]}.
\end{align}
$X_{\infty}$ は、
$X$の境界(boundary)
と呼ばれ、
$\partial X$と書く場合がある。
($X$がトロピカルトーリック多様体のとき、
$X_{\infty}$が$X$の境界つき
位相多様体の境界となっていることと、
代数幾何におけるトーリック境界の
トロピカル類似になっていることに
由来していると思われる。)
さらに言えば、$\opn{sed}_X$ は
semiconstant な関数なので、




$Y$ を $X$ 
のトロピカル部分多様体とし、
任意の点$x\in Y$に対して

\begin{align}
\opn{codim}(Y/X)\deq \dim X -\dim Y,\quad 
\opn{codim}_x(Y/X)\deq \dim_x X -\dim_x Y
\end{align}
とおくと、次が成り立つ。
\begin{align}
\opn{sed}_X(x)-\opn{sed}_Y(x)=
\opn{codim}_x(Y/X)-\opn{codim}_0(\opn{LC}_x Y/\opn{LC}_xX),
\end{align}
\begin{align}
\opn{codim}_x(Y/X) \geq 
\opn{sed}_X(x)-\opn{sed}_Y(x)\geq 0
\end{align}

とくに $\opn{codim}_x(Y/X)=1$
のときは、$\opn{sed}_X(x)-\opn{sed}_Y(x)=0$ または
$1$ である。さらに、
$\opn{sed}_X(x)-\opn{sed}_Y(x)=1$のとき、
$\dim \opn{LC}_x Y=\dim \opn{LC}_x X$
なので トロピカル部分多様体の定義より
$\opn{LC}_x Y\simeq \opn{LC}_x X$
となる。

点 $x\in X$ に対して
$\opn{lineal}(X,x)$ を$\opn{LC}_x X (\subset T_x X)$
の (maximal) lineality 空間とする。
(\cite[\textsection 2.1]{MR4246795} と
\cite[\textsection 3]{demedrano2023chern} を見よ。)



\begin{remark}
注意点として
\cite[\textsection 5]{MR3041763}の意味での
錐的多面集合の lineality 空間
は、
\cite[\textsection 2.1]{MR4246795}の意味での
極大 lineality 空間と一致する。
一致することを見るには、錐的多面集合が常に扇構造を持つことを
みれば十分である。これは、
$\mathbb{R}^{n}$ 内の凸多面集合の三角形分割を与える証明と
ほぼ同様にして証明することができる。
まず、錐的多面集合$C$は幾つかの有理凸多面錐の有限和
$C=\bigcup_{i\in I}\sigma_i$として
得ることができる。また、各有理凸多面錐は強凸と仮定して
よい。
各 $\sigma_i$ に対して$\sigma_i$を面に持つ
完備扇 $\Sigma_i$ が存在する。
(これは隅広の同変コンパクト化定理
\cite[Theorem 3]{MR337963}をトーリック幾何に適用
することで求まる。)
この共細 $\bigwedge_{i\in I}\Sigma_i$が、
$C$の多面体的錐の扇構造を与える。
この論法はコンパクトな polyhedron の
単体分割と実質的に同じである。
(例えば\cite[Theorem 2.11]{MR665919}
を見よ。)
\end{remark}

\begin{definition}[{中庸な位置}]
\label{definition-permissible-position}
$X$をトロピカル多様体とし、$Y$を$X$
トロピカル部分多様体(tropical submanifold)とする。
$Y$ が \emph{中庸な位置(moderate position)}にあるとは、
任意の$x\in Y$に対して
\begin{align}
     \opn{lineal}(Y,x) \subsetneq
 \opn{lineal}(X,x)
\end{align}
となることである。
\end{definition}

\begin{example}
\begin{enumerate}
\item $Y$が$X$上の中庸な位置にあるとき、
$Y$は$\opn{lineal}(X,x)$
が自明となる点集合上には存在しない。
\item $Y\cap X_{\mathrm{reg}}$ は、
$X_{\mathrm{reg}}$ 上の中庸な位置にある。
特に $X$ が整アフィン多様体のとき、$Y$は
常に中庸な位置にある。
\end{enumerate}

\end{example}

\begin{remark}
$Y$が$X$の中庸な位置にあるとき
補集合 $X\setminus Y$ は、大抵 finite type の条件を
満たさない。
\end{remark}

\begin{proposition}
\label{proposition-divisor-poincare}
$X$ を連結かつコンパクトな整アフィン多様体とし、
$D$ を余次元$1$のトロピカル部分多様体とする。
このとき、\cref{conjecture-rr-c-euler}と
\cref{conjecture-rr-euler}は同値である。
\end{proposition}
\begin{proof}
定義より、$D$は常に$X$上の中庸な位置にある。
また、$\opn{td}(X)=1$なので、
\begin{align}
\opn{RR}(X;-D)=(-1)^{\dim X}\opn{RR}(X;D)
\end{align}
また、$X\setminus D$は位相多様体なので、Poincar\'e 双対性より
\begin{align}
\chi_c(X\setminus D)=(-1)^{\dim X}\chi(X\setminus D)
\end{align}
が成り立つ。
\end{proof}

\begin{remark}[{特異点つき整アフィン多様体の場合}]
\label{remark-iass}
\cref{conjecture-rr-euler}や
\cref{proposition-divisor-poincare}
は、特異点つき整アフィン多様体に対して一般化できる
と期待できる。
一般に (境界を持たない) 向き付け可能な
特異点つき整アフィン多様体は、
Calabi--Yau 多様体の類似物と考えられている。
コンパクト Calabi--Yau 多様体 $M$ の標準束
$K_M$ は自明なので
Serre 双対性を考えると$M$上の因子$D$に対して
\begin{align}
\label{equation-calabi-yau-euler}
\chi(M;\mathcal{O}_M(D))=(-1)^{\dim M}
\chi(M;\mathcal{O}_{M}(-D))
\end{align}
となる。
これまでの議論を踏まえると、
\cref{equation-calabi-yau-euler} は
整アフィン多様体の場合に示した
\cref{proposition-divisor-poincare} と類似した
性質であり、\cref{proposition-divisor-poincare}
の証明は $X\setminus D$ が位相多様体であることが
本質的であるため、
\cref{proposition-divisor-poincare}
は特異点つき整アフィン多様体に一般化できると
期待できる。
\end{remark}

\begin{example}[{トロピカル射影空間}]
\label{example-TPn}
$\mathbb{T}P^{n}$ をトロピカル射影空間とし、
$F$を$d\Delta_{n}$をNewton多面体とするトロピカル
多項式とする。トロピカル超曲面 
$V_{\mathbb{T}}(F)$ が滑らかとは、$F$ が定義する正則
多面体分割が、unimodularな三角形分割になることであった。
(例えば、\cite[\textsection 4.5]{MR3287221} を
参照。)
このとき、
$\mathbb{T}P^{n}\setminus
V_{\mathbb{T}}(F)$の各連結成分が$d\Delta_n$内の格子点集合と
全単射になっていることは有名である
(例えば、\cite[Proposition 3.1.6]{MR3287221})。各連結成分は、
$\mathbb{T}P^{n}$内の凸多面体であるため、位相空間的には
すべて可縮である。また、
トーリック多様体における Chern 類の表現と
CSM サイクルによる Chern 類の定義は整合的である
\cite[Proposition 13.1.2]{MR2810322}。
よって、Todd 類も同等のものとなる
(\cite[Theorem 13.1.6]{MR2810322} も参照)。よって、
$c_1(V_{\mathbb{T}}(F))=d\in H^{1,1}(\mathbb{T}P^{n};
\mathbb{Z})
\simeq \mathbb{Z}$であることから次の等式が得られる。
\begin{align}
\opn{RR}(\mathbb{T}P^{n};V_{\mathbb{T}}(F))=
\sharp d\Delta_n(\mathbb{Z})=
\chi(\mathbb{T}P^{n}\setminus
V_{\mathbb{T}}(F))
\end{align}
なお、同様にして次の等式も得られる。
\begin{align}
\opn{RR}(\mathbb{T}P^{n};-V_{\mathbb{T}}(F))=
(-1)^{n}\sharp \opn{int}(d\Delta_n(\mathbb{Z}))=
\chi_{c}(\mathbb{T}P^{n}\setminus
V_{\mathbb{T}}(F))   
\end{align}
二番目の等式は、
$\chi_{c}(\mathbb{T}P^{n}\setminus
V_{\mathbb{T}}(F))=1-\chi (V_{\mathbb{T}}(F))$
と$V_{\mathbb{T}}(F)$が、$(n-1)$次元球面
$S^{n-1}$ 
の$\sharp \opn{int}(d\Delta_n(\mathbb{Z}))$
個の wedge 和とホモトピー同値であることから従う。
\end{example}

\begin{proposition}
$X$ をトロピカル多様体とし、$Y$ を
$X$ の余次元 $1$
のトロピカル部分多様体とする。
$Y$ が中庸な位置にあるとき、
$Y$ は sedentary 0 なトロピカル部分多様体である。
\end{proposition}
\begin{proof}
点 $x\in Y$ に対して
$\opn{sed}_X(x)-\opn{sed}_Y(x)=1$ ならば、
$\opn{LC}_xY\simeq \opn{LC}_x X$ なので
$Y$ が中庸な位置にあるとき、そのような点は
存在しない。
\end{proof}

証明には、
次のよく知られた命題を用いる。
\begin{proposition}[{Well-known}]


特に、階数 $2$ のループをもたないマトロイドが定義する
Bergman 扇が $\Gamma_n$ のいずれかと同型である。
\end{proposition}
\begin{proof}
$M$ の
Lattice of flats から単純マトロイド
$M'$ が定義される。Lattice of flats が同型ならば、
Bergman 扇の台としても同型となる。
階数 $2$ の連結な
単純マトロイドは、皆一様マトロイドである。
\end{proof}


トロピカル曲面上の各点の局所錐の構造は、
\cite[Corollary 2.4]{shaw2015tropical}
によって分類されている。
これを念頭に次の命題を解く。

\begin{proposition}
\label{proposition-self-intersection}
$S$をコンパクトトロピカル曲面とし、
$C$を$S$の部分トロピカル曲線とする。
$C$が中庸な位置にあるとき、次の等式が成り立つ。
\begin{align}
     C.C=\sharp (C_{\mathrm{sing}}\cap S_{\mathrm{reg}})
\end{align}
\end{proposition}
\begin{proof}
$C$が$S$上中庸な位置にあるとき、
$\dim \opn{lineal}(S,x)=1,2$ のみである。
$\dim \opn{lineal}(S,x)=2$ のとき、
$\opn{LC}_x C$ は、ある
$\{0,1,2\}$上の階数$2$
の単純マトロイドの
Bergman 扇の台と同型でなくてはならないので、
$\opn{val}_C(x)=2,3$ である。
次に、$\dim \opn{lineal}(S,x)=1$ のときを
考える。このとき、
$\dim \opn{lineal}(C,x)=0$である。
特に、$\opn{LC}_x C$ は、次数 $3$ 以上の木となる。
\cite[Corollary 2.4]{shaw2015tropical} 
より、
$x$の近傍は$\Gamma_m\times \mathbb{R}$の
$(m\geq 3)$の開集合の場合のみを考えればよい。

また$\opn{LC}_x C\cap \opn{lineal}(S,x)=\{0\}$である。
実際に、$\opn{LC}_x C\cap \opn{lineal}(S,x)\ne \{0\}$
ならば、$T_x C \cap \opn{LC}_x S$ は、
$\Gamma_m$と真の部分線形空間との交叉と $\mathbb{R}$
の直積となる。この凸包は $T_x C$ の全体を張らないこととなり
矛盾する。
同じ理由から、$T_x C\cap \opn{lineal}(S,x)=\{0\}$である。
このとき、$T_x C\cap \opn{LC}_x S\simeq \Gamma_m$
である。よって、バランス条件から
$\opn{LC}_x C\simeq \Gamma_m$ である。
$\opn{LC}_x C$ と $\opn{lineal}(S,x)$ はマトロイドの交叉
とみなせるので、\cite[Theorem 4.2]{MR3032930}
と随伴公式\cite[Theorem 4.11]{shaw2015tropical} より、
$C$ の自己交点数は次のように書かれる。
\begin{align}
C . C=&\opn{deg}(K_C)-K_S . C \\
=& \sum_{p\in C_{\mathrm{sing}}}(\opn{val}(p)-2)
- \sum_{p\in C_{\mathrm{sing}}\cap S_{\mathrm{sing}}}
(\opn{val}(p)-2) \\
=& \sharp (C_{\mathrm{sing}}\cap S_{\mathrm{reg}})
\end{align}

\end{proof}



\begin{proof}[{Proof of \cref{theorem-rr-euler-surface}}]

$j\colon S\setminus C\to S$ を包含写像とし、
$a_S$ を $S$ から一点空間への連続写像とする。
このとき、次の等式が成り立つ。
\begin{align}
\chi(S\setminus C)=\chi(Ra_{S*}Rj_*\mathbb{Q}_{S\setminus C}) 
\end{align}
これは、$S$上の構成可能関数となる。
$C$が$S$上中庸な位置にあるとき、
\cref{proposition-self-intersection}
の証明で見た局所錐の構造の分類より、
\begin{align}
(Rj_*\mathbb{Q}_{S\setminus C})_x
\simeq
\begin{cases}
\mathbb{Q}[0], \text{ if } x\in S\setminus C, \\
\mathbb{Q}^3[0], \text{ if } x\in C_{\mathrm{sing}}
\cap S_{\mathrm{reg}}, \\
\mathbb{Q}^2[0], \text{ otherwise.}  
\end{cases}  
\end{align}
となっている。
とくに、$Rj_*\mathbb{Q}_{S\setminus C}\simeq
j_*\mathbb{Q}_{S\setminus C}$ である。
$j^{-1}j_*=\opn{id}$ なので、
\begin{align}
0\to j_!\mathbb{Q}_{S\setminus C} 
\to j_* \mathbb{Q}_{S\setminus C}
\to (j_*\mathbb{Q}_{S\setminus C})_{C} \to 0
\end{align}
という層単完全列を持つ。
特に、
\begin{align}
\chi(S\setminus C)=
\chi_c(S\setminus C)+\chi((j_*\mathbb{Q}_{S\setminus C})|_C)
\end{align}
となる。
層の押し出しの定義より
$(j_*\mathbb{Q}_{S\setminus C})|_C$は、
$C_{\opn{reg}}$ 上では、$\mathbb{Q}^2$
を茎とする局所定数層である。
よって、
\begin{align}
\chi((j_*\mathbb{Q}_{S\setminus C})|_C)=
2\chi_c(C_{\mathrm{reg}})+
3\chi_c(C_{\mathrm{sing}}\cap S_{\mathrm{reg}})+
2\chi_c(C_{\mathrm{sing}}\cap S_{\mathrm{sing}})
\end{align}
と書かれるため、$\chi(S\setminus C)$
は次のように書くことができる。
\begin{align}
\label{equation-euler-calculus}
\chi(S\setminus C)&=
\chi_c(S\setminus C)+
3\chi_c(C_{\mathrm{sing}}
\cap S_{\mathrm{reg}})+
2\chi_c(C\setminus (C_{\mathrm{sing}}
\cap S_{\mathrm{reg}}))\\
&=\chi_c(S\setminus C)+2\chi_c(C)+
\chi_c(C_{\mathrm{sing}}
\cap S_{\mathrm{reg}}) \\
&=\chi(S)+C.C+\chi(C)
=\frac{C.(C-K_S)}{2}+\chi(S).
\end{align}
\end{proof}

\begin{remark}
\cref{theorem-rr-euler-surface} の
証明において、$S\setminus C$ の Euler 数を
計算するために、$Rj_*\mathbb{Q}_{S\setminus C}$
を定数層に分割して証明を行ったが、これは
Euler calculus と密接に関係している。
Euler calculus は構成可能層の Euler 数を計算する
上で有効な考え方であり、
\cref{equation-euler-calculus} は、
Euler 標数を測度の一種とみなし、その上で次の単調関数を
積分していると解釈することができる。
\begin{align}
\chi(S\setminus C)(x)\deq \chi(Rj_*\mathbb{Q}_{S\setminus C})_x 
\simeq \chi(\varinjlim_{U\ni x} H^{\bullet}((S\setminus C)
\cap U;\mathbb{Q}))   
\end{align}
Euler calculus については、例えば\cite{}などを見よ。
\end{remark}

\begin{example}
$\opn{deg}(f)=e$, $\opn{deg}(g)=d$とする。
\begin{align}
(V(g)|_{V(f)})^{2}+K_{V(f)}.(V(g)|_{V(f)})
=d^2e+(e-4)ed=d^2e+de^2-4ed
\end{align}

\end{example}

\section{その他の例}

ここでは、トロピカル多様体のTodd類が
満たすべき性質を持っている場合
いくつかの初等的な例で予想が成立している
ことを示す。
トロピカル多様体の Chern 類は、ベクトル束の
理論の直接的な類似として定義されていないため、
現時点では多くのことはわかっていない。
例えば、全 Chern 類に関する
分裂原理(splitting principle)周りの
議論のトロピカル類似がよくわかっていない。
一方で乗法列\cite[\textsection 1]{MR1335917}の
定義および
形式的な議論のみで完了する内容はうまくいく
部分がある。
ここでは形式的な議論でうまくいく部分と
Todd 類が持つべき性質を仮定した場合に、
\cref{conjecture-rr-bertini} 
が\cite[Conjecture 6.13]{demedrano2023chern}
と\cref{conjecture-rr-euler} より
帰結されることを見る。



\subsection{トロピカルホモロジー側の準備}

ここでは、トロピカル多様体 
(より一般的には、滑らかな有理多面空間
\cite[Definition 6.1]{MR4637248})
の層理論的なトロピカルホモロジー論について、
必要な部分を \cite{MR4637248} から
復習する。

$(X,\mathcal{O}_X^{\times})$ を有理多面空間
とし、$\mathcal{M}_X^{\times}$ を
$X$ 上の$\mathbb{Z}$ 係数の傾きをもつ区分
整アフィンな関数のなす層とする
(\cite[Definition 4.1]{MR3894860}
や\cite[Definition 3.8 and Remark
3.9]{MR4246795}を見よ)。

\begin{remark}
トロピカル多様体上の有理関数の定義は、著者によって
異なる。とくにそれに応じて、
トロピカル多様体上の Cartier 因子の定義も異なってくる。
例えば、\cite{MR3894860,MR4637248}
における Cartier 因子の台は、
すべてのトロピカルサイクルの台を表現するとは限らない。
$X\deq \mathbb{T}$ としたとき、$\{-\infty\}$
は、\cite{MR3894860,MR4637248} の意味での
Cartier 因子の台にはならない。
\end{remark}


$n$ 次元有理多面空間
$(X,\mathcal{O}_X^{\times})$ が、
基本類を許す
(admits a fundamental class)
とは、$X$ が純次元的であり、
$X$ 上の$1$を値にもつ定数関数が
$Z_n(X)$ の元を定義することである
\cite[\textsection 6.1]{MR4246795}。


\subsection{トロピカル多様体の Todd 類}

古典的には、複素多様体 $M$ 上の偶コホモロジー環
$H^{\mathrm{even}}(M;\mathbb{Q})$
と接束 $TM$の全 Chern 類 
$c(TM)\deq \sum_{i=0}^{\infty}c_i(TM)\in 
H^{\mathrm{even}}(X;\mathbb{Q})$
に対して
Todd $m$-列 $(\opn{Todd}_j)_{j\in \mathbb{Z}_{\geq 0}}$
を適用させることで Todd 類
$\opn{td}(X)\deq \sum_{j=0}^{\infty}
\opn{Todd}_j(c_1(TM),\ldots,c_j(TM))\in 
H^{\mathrm{even}}(M;\mathbb{Q})$
が定義されるのであった
\cite[\textsection 10]{MR1335917}。

上述の$H^{\mathrm{even}}(M;\mathbb{Q})$を
$\bigoplus_{i=0}^{\infty} H^{i,i}(X;\mathbb{Q})$
に置き換えるだけで同様の議論ができる。

トロピカル多様体$X$にも同様に
$\opn{cyc}_k\colon Z_k(X)\to H^{\mathrm{BM}}_{k,k}(X;\mathbb{Z})$
が存在する
\cite{gross2019sheaftheoretic}。
(\cite[Definition 4.13]{MR3894860}もみよ。)
また、$X$が純 $n$ 次元トロピカル多様体のとき、
Poincar\'e 双対
$\opn{PD}_X\colon H_{k,k}^{\opn{BM}}(X;\mathbb{Z})
\to H_{n-k,n-k}^{\opn{BM}}(X;\mathbb{Z})$
も存在する。この準同型射はトロピカル多様体の
基本類のカップ積の射 \eqref{equation-cup-pd}
の逆写像である。

\begin{notation}
$X$ を純 $n$ 次元トロピカル多様体としたとき、
$X$ の 第 $k$ Chern 類を次のように書く。 
\begin{align}
c_{k}^{\mathrm{sm}}(X)\deq
\opn{PD}_X\circ \opn{cyc}_{n-k}(\opn{csm}_{n-k}(X))
\in H^{k,k}(X;\mathbb{Z})
\end{align}
なお、$c_{k}^{\mathrm{sm}}(X)$と書いたのは、
因子類の Chern 類との混同を避けるためである。
同様に$X$の全Chern類を
$c^{\mathrm{sm}}(X)=\sum_{k=0}^{n} c_{k}^{\mathrm{sm}}(X)$
とする。
\end{notation}
\cite{demedrano2023chern}に従いTodd類を次のように
定義する。
\begin{definition}[{\cite{demedrano2023chern}}]
$X$をトロピカル多様体とする。
第 $j$ 次 Todd 類を
$\bigoplus_{i=0}^{\infty} H^{i,i}(X;\mathbb{Q})$に
関する Todd $m$-列 $\opn{Todd}\deq (\opn{Todd}_j)_{j\in \mathbb{Z}_{\geq 0}}$
に対して
$X$ の全 Chern 類
$c^{\mathrm{sm}}(X)$を代入することで得られるものとする。
\begin{align}
\opn{td}_j(X)\deq \opn{Todd}_j(c_{1}^{\mathrm{sm}}(X),
\ldots,c_{j}^{\mathrm{sm}}(X)),
\end{align}
\begin{align}
\opn{td}(X)\deq \sum_{j=0}^{\infty}\opn{td}_j(X)=
\opn{Todd}(c^{\mathrm{sm}}(X)).
\end{align}

\end{definition}

$\pi_1\colon X_1\times X_2 \to X_1$

\begin{align}
Ra_{X*}
\end{align}

\begin{proposition}
$X_1,X_2$ をトロピカル多様体とし、
\begin{align}
c^{\mathrm{sm}}(X_1\times X_2)=
\pi^{*}_1(c^{\mathrm{sm}}(X_1))\cdot 
\pi^{*}_2(c^{\mathrm{sm}}(X_2))
\end{align}

\end{proposition}

\begin{proof}
トロピカルサイクルと
トロピカル
Borel--Moore ホモロジーの
クロス積の定義
\cite[Definition 3.7, 
Definition 4.15]{gross2019sheaftheoretic}
とその整合性
\cite[Proposition 5.9]{gross2019sheaftheoretic}
および Poincar\'e 双対性より
Chern--Schwartz--Macpherson cycleに関する
次の等式を示せばよい。
\begin{align}
\opn{csm}_k(X_1\times X_2)=\sum_{i+j=k}
\opn{csm}_{i}(X_1)\times \opn{csm}_{j}(X_2)
\end{align}
これは、
$\pi_1^{*}\mathcal{F}^{\bullet}_{\mathbb{Z},X_1}\otimes
\pi_2^{*}\mathcal{F}^{\bullet}_{\mathbb{Z},X_2}
\simeq \mathcal{F}^{\bullet}_{\mathbb{Z},X_1\times X_2}$
であること\cite[Lemma 4.14]{gross2019sheaftheoretic}
と
$X^{[*,k]}\deq \{x\in X\mid \dim \opn{lineal}(X,x)=k\}$
に関する次の公式からわかる。
\begin{align}
(X_1\times X_2)^{[*,k]}=\bigsqcup_{i+j=k}
X_1^{[*,i]}\times X_2^{[*,j]}
\end{align}

\end{proof}

\begin{remark}上述の命題は
\cite[Proposition 5.1]{demedrano2023chern}
からも証明できる。
\end{remark}

よって、次の Todd 類の乗法性と
因子$D$の Riemann--Roch 数に対する
 K\"unneth の公式が成り立つ。

\begin{proposition}
$X_1,X_2$をトロピカル多様体とする。
\begin{align}
\opn{td}(X_1\times X_2)=\pi_{1}^{*}\opn{td}(X_1)\cdot 
\pi_{2}^{*}\opn{td}(X_2)
\end{align}
とくに、任意の因子$D_i\in \opn{CDiv}(X_i)$ ($i=1,2$)
に対して次が成り立つ。
\begin{align}
\opn{RR}(X_1\times X_2;D_1\boxtimes D_2)
=\opn{RR}(X_1;D_1)\opn{RR}(X_2;D_2)
\end{align}

\end{proposition}

\begin{proof}
Todd $m$-列の乗法性より次が成り立つ。
\begin{align}
\opn{td}(X_1\times X_2)
&=\opn{Todd}(\pi^{*}_1(c^{\mathrm{sm}}(X_1)))
\cdot \opn{Todd}(\pi^{*}_2(c^{\mathrm{sm}}(X_2))) \\
&=\pi^{*}_1\opn{td}(X_1)\cdot 
\pi^{*}_2\opn{td}(X_2).
\end{align}
よって、$\opn{ch}(\pi_1^{*}D_1+\pi_2^{*}D_2)=
\opn{ch}(\pi_1^{*}D_1)\cdot \opn{ch}(\pi_2^{*}D_2)$
より、
\begin{align}
\opn{ch}(\pi_1^{*}D_1+\pi_2^{*}D_2)
\pi^{*}_1\opn{td}(X_1)\cdot 
\pi^{*}_2\opn{td}(X_2)=
\pi_1^{*}(\opn{ch}(D_1)\opn{td}(X_1))\cdot
\pi_2^{*}(\opn{ch}(D_2)\opn{td}(X_2)).
\end{align}

\end{proof}

\begin{proposition}

\end{proposition}
\begin{proof}

\begin{align}
\opn{RR}(X_1;D_1)
\opn{RR}(X_2;-D_2)
= &\chi(X_1\setminus D_1)\chi_c(X_2\setminus D_2) \\
= &\chi(X_1\setminus D_1)
(\chi(X_2)-\chi(D_2)) \\
= &\chi(X_1\times X_2\setminus D_1\times X_2)
-\chi(D_1\times X_2\setminus D_1\times D_2)
\end{align}

\end{proof}




\begin{example}
$X_1,X_2$ をトロピカル多様体とし 
$D_1,D_2$ を中庸な位置にある
余次元$1$のトロピカル部分多様体とする。
このとき、
\begin{align}
\chi(X_1\setminus D_1)\chi(X_2\setminus D_2)
=\chi(X_1\times X_2 \setminus (D_1\times X_2\cup X_1\times D_2)) \\
=\chi(X_1\times X_2 \setminus (\pi_1^{*}D_1+\pi_2^{*}D_2))
\end{align}

\end{example}

ここで代数多様体 $X$ の接束を$\mathcal{T}_X$とすると、
非特異因子 $D$ に対して、次の層短完全列が存在する。
\begin{align}
0 \to \mathcal{T}_{D}\to \iota^{*}\mathcal{T}_X
\to \mathcal{N}_{D/X}\to 0
\end{align}
ここで、$\mathcal{N}_{D/X}\simeq \iota^{*}\mathcal{O}_X(D)$
であるため、次の全 Chern 類の同伴公式が成り立つ。
\begin{align}
\label{equation-classical-total-adjunction}
\iota^{*}c(\mathcal{T}_X)
=c(\mathcal{T}_{D})c(\iota^{*}\mathcal{O}_X(D))
\end{align}
次数部分に着目すれば、$k\in \mathbb{Z}_{\geq 0}$
に対して
\begin{align}
\label{equation-classical-total-adjunction}
\iota^{*}c_k(\mathcal{T}_X)
=c_{k}(\mathcal{T}_{D})+
c_{k-1}(\mathcal{T}_{D})c_1(\iota^{*}\mathcal{O}_X(D))
\end{align}
である。
なお、Chern 類は、Chow 群の元としても定義でき、同様の
同伴公式が成立する。
\eqref{equation-classical-total-adjunction}の
トロピカル幾何的類似は考察でき、次の予想が期待できる。
\begin{conjecture}
\label{conjecture-grr-divisor}
$X$をトロピカル多様体とし、$D$ を余次元$1$の
部分トロピカル多様体とする。このとき、次が成り立つ。
\begin{align}
\label{equation-total-adjunction}
\iota^{*}c^{\mathrm{sm}}(X)=c^{\mathrm{sm}}(D)c(\iota^{*}[D]), \\ 
\iota^{*}c^{\mathrm{sm}}_k(X)=c^{\mathrm{sm}}_k(D)+c^{\mathrm{sm}}_{k-1}(D)\iota^{*}[D]
\end{align}
ここで、$c(\iota^{*}[D])\deq 1+c_1(\iota^{*}[D])$である。 
\end{conjecture}
\cref{conjecture-grr-divisor}では、簡単のために
トロピカルコホモロジーの元に対するものとしたが、
トロピカル Chow 群版も同様に期待できる。
$\dim X=1,2$ のとき、
\cref{conjecture-grr-divisor}
は正しい。$\dim X=1$のときは、
明らかである。
$\dim X=2$のときは、
トロピカル曲面内の
部分トロピカル曲線の
同伴公式の別表現に過ぎない
\cite[Theorem 6]{shaw2015tropical}。
より詳しくは、\cref{example-gysin-surface}
にて記述する。
\cref{conjecture-grr-divisor} 
が正しいならば、
どのような結論を導くのか見てみる。

$X$が純 $n$ 次元トロピカル多様体のとき、
任意の$p\in\mathbb{Z}_{\geq 0}$に対して
次の同型が存在する
\cite[Theorem 6.2]{gross2019sheaftheoretic}。

\begin{align}
\opn{PVD}^{(n-p)}_X\colon \Omega_{X}^{n-p}[n]\simeq
\mathcal{D}_{\mathbb{Z}_X}(\Omega_X^{p})
\end{align}
とくに、$[X]\deq \opn{PVD}^{(n)}_X\in H_{n,n}^{\mathrm{BM}}
(X;\mathbb{Z})$
はトロピカル多様体の基本類(fundamental class)
である (cf. \cite[Definition 4.8]{MR3894860})。
また、$\opn{PVD}^{(n-p)}_X$は、射の合成により同型
\begin{align}
H^{n-p,n-q}(X;\mathbb{Z})& \simeq
\opn{Hom}_{D^{b}(\mathbb{Z}_X)}
(\mathbb{Z}_X,\Omega_X^{n-p}[n-q]) \\
& \simeq \opn{Hom}_{D^{b}(\mathbb{Z}_X)}
(\mathbb{Z}_X,\mathcal{D}_X(\Omega_X^{p})[-q])
\label{equation-pvd-rhom} \\
& \simeq \opn{Hom}_{D^{b}(\mathbb{Z}_X)}
(\Omega_X^{p}[q],\omega_X^{\bullet})
\simeq H_{p,q}^{\mathrm{BM}}(X;\mathbb{Z})
\label{equation-pvd-tensor}
\end{align}
を与える。さらに
\cref{equation-pvd-rhom,equation-pvd-tensor}
の同型は tensor-hom 随伴から由来するため、
射の合成
\begin{align}
\mathbb{Z}_X \xto{\alpha} \Omega_X^{n-p}[n-q] 
\xto{\opn{PVD}^{(n-p)}_X} 
R\mathcal{H}om (\Omega_X^{p},\omega_X^{\bullet}[-q])
\end{align}
は、tensor-hom 随伴より次の射の合成
\begin{align}
\Omega_X^{p}[q]\xto{\alpha \otimes \opn{id}} \Omega_X^{n-p}[n-q]\otimes^{L}_{\mathbb{Z}_X}
\Omega_X^{p}[q] \to\omega_X^{\bullet}
\end{align}
に対応する。
さらに、$\opn{PVD}^{(n-p)}_X$の射の構成と
Borel--Moore ホモロジーのカップ積の構成を
先ほどと同様に tensor-hom 随伴と導来関手の
普遍性に注意しながら見ると、
$\opn{PVD}^{(n-p)}_X$ は、
基本類のカップ積による同型
\begin{align}
\label{equation-cup-pd}
\cdot \frown [X]
\colon H^{n-p,n-q}(X;\mathbb{Z})\simeq
H^{\mathrm{BM}}_{p,q}(X;\mathbb{Z})
\end{align}
を与えていることがわかる。
カップ積 $\cdot \frown [X]$
の逆準同型を$\opn{PD}_X$ とする。

($X$ が global な面構造を持つ場合の、
基本類のカップ積による上述の同型は、
\cite[Theorem 5.3]{MR3894860} にて初めて示された。
また Poincar\'e 双対性は
\cite[Theorem 1.2]{amini2021homology}によって、
より広いクラス(彼らの意味での smooth の定義)
に一般化された。なお、トロピカル多様体(tropical variety)
が、Verdier--Poincar\'e 双対性を持つときの
必要十分条件は、\cite[Theorem 6.7]{MR4637248}
を見よ。)

次に、古典の場合に従い
一般化された Gysin 押し出し(generalized Gysin map)
を定義する(e.g. 
\cite[Chapter 13. Appendix]{MR2810322})。
(なお、トロピカル多様体における
一般化された Gysin 押し出しは
\cite[Proposition 8.3]{} でも用いられている。)
\begin{definition}
$X,Y$ を pure なトロピカル多様体とし、
$Q$を$\mathbb{R}$の部分環とする。
$f\colon X\to Y$ を固有射とする。
$f$ の \emph{generalized Gysin map} とは、
次の射が可換となる射
$f_!\colon H^{\bullet,\bullet}(X;Q)\to 
H^{\bullet,\bullet}(Y;Q)$
のことである。
\begin{equation}
\begin{tikzcd}
	{H^{\bullet,\bullet}(X;Q)} & {H^{\bullet,\bullet}(Y;Q)} \\
	{H_{\bullet,\bullet}^{\mathrm{BM}}(X;Q)} & {H_{\bullet,\bullet}^{\mathrm{BM}}(Y;Q)}
	\arrow["{\cdot \frown [X]}"', from=1-1, to=2-1]
	\arrow["{\cdot\frown[Y]}", from=1-2, to=2-2]
	\arrow["{f_*}"', from=2-1, to=2-2]
	\arrow["{f_!}", from=1-1, to=1-2]
\end{tikzcd}    
\end{equation}

\end{definition}
定義より、次の射影公式が成り立つ。
\begin{proposition}[{射影公式}]
$X,Y$ を pure なトロピカル多様体とし、
$f\colon X\to Y$ を固有射とする。
このとき、$c\in H^{\bullet,\bullet}(Y;Q) $
$d\in H^{\bullet,\bullet}(X;Q)$
に対して、次が成り立つ。
\begin{align}
    f_!(f^{*}(c)\cdot d)=c\cdot f_!(d).
\end{align}
とくに、$f_!(f^{*}(c))=c\cdot f_!(1)$ が成り立つ。
\end{proposition}

\begin{proof}
$\cdot \frown [Y]$ は同型射なので、次の等式を
示せばよい。
\begin{align}
    f_!(f^{*}(c)\cdot d)\frown [Y]=(c\cdot f_!(d))\frown [Y]
\end{align}
実際に、
\begin{align}
(c\cdot f_!(d))\frown [Y]=
c\frown(f_!(d)\frown [Y])=
c\frown f_*(d\frown [X])\\
=f_*(f^*(c)\frown (d\frown [X]))=
f_*((f^*(c)\cdot d) \frown [X])
=f_!(f^{*}(c)\cdot d)\frown [Y].
\end{align}
よって示せた。
\end{proof}

\begin{example}
     
$X$ をトロピカル多様体とし $D$ を余次元 $1$ の
部分トロピカル多様体とする。
まず、$\iota\colon D\to X$ に関する
Gysin 押し出しと射影公式から
\begin{align}
c^{\mathrm{sm}}(X)\cdot [D]=\iota_!c^{\mathrm{sm}}(D)c([D]).
\end{align}

この等式は、第一次数部分に着目すれば、
\cite[Theorem 5.2]{demedrano2023chern}にて示された
同伴公式の高次元版の一種とみなせる。

一方で\eqref{equation-total-adjunction}と
Todd類の性質より
\begin{align}
\iota^{*}\opn{td}(X)=
\opn{td}(D)\frac{\iota^{*}[D]}{1-\opn{exp}(\iota^{*}[D])}.
\end{align}
これを式変形すると次のようになる。
\begin{align}
\opn{td}(D)
=\iota^{*}((1-\opn{exp}([D])/[D])\opn{td}(X))
\end{align}
また、これに射影公式
\cite[Proposition 4.18]{gross2019sheaftheoretic}
を当てはめると
\begin{align}
\label{equation-grr-divisor}
\iota_!(\opn{td}(D))=(1-\opn{ch}(D))\opn{td}(X).
\end{align}
\eqref{equation-grr-divisor} は、
Grothendieck--Riemann--Roch の定理の特殊例
のトロピカル類似である。

\end{example}

以上を踏まえて、\cref{theorem-rr-euler-surface}
を一般化する。

\begin{theorem}
\label{theorem-rr-bertini-surface}
$X$を
コンパクトトロピカル曲面とし、$D$ と $D'$が余次元$1$
のトロピカル部分多様体とする。

\begin{align}
\chi(X\setminus D')-\chi(D\setminus (D'\cap D))
=\frac{(D'-D).(D'-D-K_X)}{2}+\chi(X).
\end{align}
特に $X$ が Delzant 面構造をもつとき、
\cref{conjecture-rr-bertini} は正しい。
\end{theorem}

\begin{proof}
$\dim X=2$なので、
\cref{conjecture-grr-divisor}は正しい。
そのため射影公式と\eqref{equation-grr-divisor}より、
\begin{align}
\iota_!(\opn{ch}(\iota^{*}(D'))\opn{td}(D))
=\opn{ch}(D')\iota_!(\opn{td}(D))
=(\opn{ch}(D')-\opn{ch}(D'-D))\opn{td}(X).
\end{align}

よって、次が成り立つ。
\begin{align}
\label{equation-rr-number-divisor}
\opn{RR}(X;D'-D)=\opn{RR}(X;D')-
\opn{RR}(D;\iota^{*}(D')).
\end{align}

\cref{theorem-rr-euler-surface}と
\eqref{equation-rr-number-divisor}より、
\begin{align}
\frac{(D'-D).(D'-D-K_X)}{2}=&
\opn{RR}(X;D')-\opn{RR}(X;0)-
\opn{RR}(D;D'\cap D) \\
=&\chi(X\setminus D')-\chi(X)-\chi(D\setminus (D'\cap D)).
\end{align}
\end{proof}

\begin{remark}
\cite[Conjecture 6.13]{demedrano2023chern},
\cref{conjecture-grr-divisor,conjecture-rr-euler}
が正しいなら、\cref{conjecture-rr-bertini}も正しい。  
\end{remark}

\begin{proof}





\end{proof}



\section{補集合と局所 Morse データ}

\begin{question}
Berkovich curve の II, III型の点とこの補空間の理論は
関係があるのではなかろうか? 
(type Iの点とはおそらく相性が悪い。超平面配置の解析化が
可縮であることからも想像できる。)
\end{question}





\subsection{*因子の話(文中のどこかに書く話)}
この節では、古典的な代数曲面の交点数の異なる定義
の一致をトロピカル曲面の場合で見る。
\begin{proposition}
\label{proposition-cycle-chern}
$X$ をコンパクトトロピカル曲面とし、$D_1,D_2$
をトロピカル$1$-サイクルとする。このとき、
次が成り立つ。
\begin{align}
D_1 . D_2=\int_X c_1(\mathcal{L}(D_1))
\cdot c_1(\mathcal{L}(D_2))
\end{align}

とくに右辺は、古典的なトロピカルホモロジーの
交叉積と一致する。

\end{proposition}

まず、トロピカル多様体の余次元 $1$ のトロピカルサイクル
は、Cartier 因子による表現を持つ
\cite[Proposition 3.27]{shaw2015tropical}。
なお、本文書でのトロピカル多様体上の 
Cartier 因子の定義は、
\cite{shaw2015tropical} や
\cite{demedrano2023chern} などで用いられる
ものである点に注意せよ。(Mikhalkin の定義に遡る。)
また、Cartier 因子から直線束
$\mathcal{L}(D)\in H^{1}(X;\mathcal{O}_X^{\times})$
が定義される。

$(X,\mathcal{O}_X^{\times})$ を有理多面空間
とし、$\mathcal{M}_X^{\times}$ を
$X$ 上の$\mathbb{Z}$ 係数の傾きをもつ区分
整アフィンな関数のなす層とする。
さらに $(X,\mathcal{O}_X^{\times})$ がトロピカル
多様体のとき、
$\mathcal{K}_{X}^{\times}
\deq \iota_*\mathcal{M}_{X^{[0]}}^{\times}$
とする。
$\mathcal{D}iv_X\deq \mathcal{M}_X^{\times}/
\mathcal{O}_X^{\times}$、
$\opn{Div}(X)^{[0]}\deq \Gamma(X;\mathcal{M}_X^{\times}/
\mathcal{O}_X^{\times})$
とする\cite[Deifinition 3.10]{MR4637248}。
$\opn{Div}(X)^{[0]}$ は、
\cite{demedrano2023chern} などの意味の Cartier 
因子のなす群 $\opn{Div}(X)$ の部分群である。

次に、$Z_k(X)$ を $X$ 上のトロピカル $k$-サイクルの
なす群とする\cite[Definition 3.5]{MR4637248}。
$X$の開集合に対して$Z_k(U)$を値に持つ前層は層をなし、
これを$\mathscr{Z}_k^{X}$と書く。
各 $Z_k(X)$ にはサイクル写像
$\opn{cyc}_X \colon Z_k(X)\to 
H^{\mathrm{BM}}_{k,k}(X;\mathbb{Z})$
が存在し、固有射の押し出しに関して可換である
\cite[Definition 5.4 and Corollary 5.8]{MR4637248}。
また、次のようなペアリングが存在する
\cite[\textsection 3.4]{MR4637248}。
\begin{align}
\label{equation-divisor-pairing}
\opn{Div}(X)^{[0]}\times Z_{k}(X)\to Z_{k-1}(X);
(D,A) \mapsto D\cdot A
\end{align}

これは、次のように作られている。

次に、有理凸多面空間の射$f\colon X\to Y$はスキームの場合と異なり、
任意の因子 $D$ の引き戻し $f^{*}D$ が常に存在し、
付随する直線束 $\mathcal{L}(D)$ の引き戻しと整合的である
ことに注意する
\cite[Propoisition 3.15]{gross2019sheaftheoretic}。
よって、次のような射が定義できる。

\begin{align}
\opn{Div}(X)^{[0]}\times Z_{k}(|A|) \to 
\opn{Div}(|A|)^{[0]} \times Z_{k}(|A|) \to
Z_{k-1}(|A|) \to
Z_{k-1}(X)
\end{align}

そして、$X$内のトロピカルサイクルの台となる
部分空間の包含射による押し出し
による余極限をとることで
\cref{equation-divisor-pairing} が得られる。


とくに次のような等式が成り立つ
\cite[Proposition 5.12]{MR4637248}。
(\cite[Theorem 4.15]{MR3894860}もみよ。)
\begin{align}
\opn{cyc}_X(D\cdot A)=c_1(\mathcal{L}(D))
\frown \opn{cyc}_X(A).
\end{align}

$X$ が純 $n$ 次元トロピカル多様体であるとき、
$X_{\mathrm{reg}}$ 上の値を$1$とする定数関数
$1_{X_{\mathrm{reg}}}$が、$X$上のトロピカル
$n$-サイクルとなり、
$\opn{cyc}_X(1_{X_{\mathrm{reg}}})=[X]$
となる \cite[\textsection 6.1]{MR4637248}。
$X$ が連結かつコンパクトな $n$ 次元トロピカル多様体とし、
$Y$ を $X$ の余次元 $1$ の
トロピカル部分多様体とする。
このとき、
$\iota_*(1_{Y_{\opn{reg}}})\in Z_{n-1}(X)$ に対して
$D\cdot 1_{X_{\mathrm{reg}}}=\iota_*(1_{Y_{\opn{reg}}})$
となる Cartier 因子 $D$ が存在する。
この Cartier 因子が一意であることは、
次のことからわかる。
まず、ペアリングの定義が局所的なので
次の層の射が誘導される。
\begin{align}
\opn{div}_X\colon \mathcal{D}iv_X\to 
\mathscr{Z}_{n-1}^{X}
\end{align}
この層が単射であることを示せばよいため、
$X=L_M\times \underline{\mathbb{R}}^{n}$ とし、
$x=(0,\ldots,0,-\infty,\ldots,-\infty)$と
仮定してよい。
$f\colon L_M\times \underline{\mathbb{R}}^{n}\to 
\mathbb{R}$ をZPL関数とすると
$x$の十分に小さい近傍上では、
$f$ は、$L_M$上のZPL関数の引き戻しで
得られるものだとわかる。また、
$\opn{div}_X(f)$ で定義されるサイクルも
$L_M$上のサイクルの引き戻しである。
よって、$X=L_M$ の場合を調べれば
よいが、これは
\cite[Theorem 4.5]{MR4246795}
の証明の中でより一般的に示されている。

以降、余次元 $1$ の部分トロピカル多様体と
それに付随する Cartier 因子は同一視する。

\begin{align}
\iota_*[D]=c_1(\mathcal{L}(D))\frown [X], \quad
\opn{PD}_X(\iota_*[D])=c_1(\mathcal{L}(D)).
\end{align}

ここで

\begin{align}
(\opn{PD}_X(\iota_*[D])\cdot \opn{PD}_X(\iota_*'[D']))\frown [X]=
c_1(\mathcal{L}(D'))\frown \iota_*[D]
\end{align}

\begin{proof}[{Proof of \cref{proposition-cycle-chern}}]

$D_1,D_2$ は、 sedentary 0 であると仮定してよい。

$D_i'$ ($i=1,2$) を $D_i$を表す Cartier 因子とすると、

\begin{align}
\opn{cyc}_X(D_i)=c_1(\mathcal{L}(D_i'))\frown [X], \quad
\opn{PD}_X(\opn{cyc}_X(D_i))=c_1(\mathcal{L}(D'_i)).
\end{align}
となる。

\begin{align}
D_1\cdot_{\opn{PD}} D_2
\deq \int_X \opn{PD}_X(\opn{cyc}_X(D_1))\cdot \opn{PD}_X(\opn{cyc}_X(D_2))\\
=\int_X c_1(\mathcal{L}(D'_1))\cdot c_1(\mathcal{L}(D'_2))
\end{align}

また、

\begin{align}
\int_X \opn{PD}_X(\opn{cyc}_X(D_1))\cdot 
\opn{PD}_X(\opn{cyc}_X(D_2))& =
a_{X*} (c_1(\mathcal{L}(D_1')) \frown \opn{cyc}_X(D_2)) \\
& =a_{X*}(\opn{cyc}_X(D_1'\cdot D_2)) 
\label{equation-cartier-weil-intersection}
\end{align}
$D_1,D_2$ は、sedentary 0 であると仮定したので、
\cref{equation-cartier-weil-intersection}
は、\cite{shaw2015tropical,demedrano2023chern}
で言及されているように$D_1$と$D_2$の交点数となる。

\begin{align}
\opn{cyc}_X(D_1'\cdot D_2')=c_1(\mathcal{L}(D_1'))
\end{align}
となる。
\end{proof}


\bibliography{tropical-complement}
\bibliographystyle{amsalpha}

\end{document}