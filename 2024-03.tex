\documentclass[dvipdfmx,12pt]{beamer}%handoutを付け加えることでスライドのアニメを消せる

% \usepackage{tikz-cd}

\usepackage{hyperref}
\usepackage{bm}%斜体ボルド(太字)用 \bm{}を用いればよい
%\usepackage{mathabx}%大型演算子補強
\usepackage{makecell}%表のセル内で改行するためのコマンド
\usepackage{bxdpx-beamer}
\usepackage{pxjahyper}
\usepackage{layout}
\usepackage{amstext,amsfonts,amssymb,amscd,
amsbsy,amsmath}

\usepackage{mathrsfs}
\usepackage{bookmark}
\usepackage[T1]{fontenc}
\usepackage{tikz-cd}
\usepackage{latexsym}%数式用
\usepackage{mathtools}%coloneqqを使うため

 \usetikzlibrary{cd}
 \usetikzlibrary{calc}

\newtheorem{question}[theorem]{Question}
\newtheorem{conjecture}[theorem]{Conjecture}
\newtheorem{proposition}[theorem]{Proposition}
\theoremstyle{definition}
\newtheorem{remark}[theorem]{Remark}

\renewcommand{\kanjifamilydefault}{\gtdefault}% 既定をゴシック体に
\setbeamertemplate{navigation symbols}{}%下の方にある変なのを消すために必要
\setbeamercovered{transparent}%隠している文章を半透明化
%\usepackage{minjis}%min10をなくせるらしい?

\usetheme{Berlin}
\usecolortheme{rose}
%\usetheme{Darmstadt}%自分の基本型
%\usetheme{Berlin}%基本型の一種
%\usetheme{boxes}
\usepackage{adjustbox}
\setbeamerfont{frametitle}{size=\normalsize}
\setbeamertemplate{frametitle}{%
    \nointerlineskip%
    \begin{beamercolorbox}[wd=\paperwidth,ht=2.5ex,dp=1.0ex]{frametitle}
        \hspace*{1ex}\insertframetitle%
    \end{beamercolorbox}%
}
%フラットデザイン化
\setbeamertemplate{blocks}[rounded] % Blockの影を消す
\useinnertheme{rectangles} % 箇条書きをシンプルに
\setbeamertemplate{headline}
{
    \begin{beamercolorbox}[ht=2.25ex,dp=1ex,center]{author in head/foot}%
      \end{beamercolorbox}   
}
%\setbeamertemplate{footline}[frame number] % フッターはスライド番号のみ

\setbeamertemplate{footline}
{
  \leavevmode%
  \hbox{%
    \begin{beamercolorbox}[wd=.25\paperwidth,ht=2.5ex,dp=1ex,center]{author in head/foot}%
      \usebeamerfont{author in head/foot}\insertshortauthor~~(\insertshortinstitute)
    \end{beamercolorbox}%
    \begin{beamercolorbox}[wd=.5\paperwidth,ht=2.5ex,dp=1ex,center]{title in head/foot}%
      \usebeamerfont{title in head/foot}\insertshorttitle
    \end{beamercolorbox}%
    \begin{beamercolorbox}[wd=.25\paperwidth,ht=2.5ex,dp=1ex,right]{date in head/foot}%
      \usebeamerfont{date in head/foot}\insertframenumber/\inserttotalframenumber
\hspace*{2em}
      \usebeamertemplate{page number in head/foot}\hspace*{2ex} 
    \end{beamercolorbox}
  }%
  \vskip0pt%
}

%Beamer色設定
\definecolor{UniBlue}{RGB}{0,150,200} 
\definecolor{AlertOrange}{RGB}{255,76,0}
\definecolor{AlmostBlack}{RGB}{38,38,38}
\colorlet{myblue}{blue!70!black} %
\setbeamercolor{normal text}{fg=AlmostBlack}  % 本文カラー
\setbeamercolor{structure}{fg=UniBlue} % 見出しカラー
\setbeamercolor{block title}{fg=UniBlue!50!black} % ブロック部分タイトルカラー
\setbeamercolor{alerted text}{fg=AlertOrange} % \alert 文字カラー

\mode<beamer>{
    \definecolor{BackGroundGray}{RGB}{254,254,254}
    \setbeamercolor{background canvas}{bg=BackGroundGray} % スライドモードのみ背景をわずかにグレーにする
}

\setbeamertemplate{theorems}[numbered]

% \newcommand{\deq}{\coloneqq}
% \newcommand{\textalert}[1]{{\normalsize \color{blue!70!black} {#1}}}
\usepackage{relsize}
% \newcommand{\mathalert}[1]{{\color{blue!70!black}\mathlarger{#1}}}

% \newcommand{\simto}{ 
% \mathrel{\raisebox{0.13em}{${\sim}$}}
% \kern -0.75em \mathrel{\raisebox{-0.11em}{${\scriptstyle \to}$}}  
% }
%\renewcommand{\kanjifamilydefault}{\gtdefault}
\usepackage[deluxe, expert, uplatex]{otf}% textbfを太くする.

\newcommand{\opn}[1]{\operatorname{#1}}
\newcommand{\deq}{\coloneqq}

\title[{The complement of a tropical 
hypersurface}]{The complement of a tropical 
hypersurface in permissible position (in progress)} 
\author[Y. Tsutsui]{Yuki Tsutsui}
\institute[UT]{Graduate School of Mathematical Sciences \and University of Tokyo}
\date{6th Tropical Geometry Workshop,\\

2024-03-14}

\begin{document}

\setlength{\abovedisplayskip}{3pt}
\setlength{\belowdisplayskip}{3pt}

\maketitle

\begin{frame}
\frametitle{Preliminary}
\end{frame}

\begin{frame}


\begin{conjecture}[{de Medrano--Rinc\'on--Shaw'23}]
For any cpt tropical mfd $S$, 
\begin{align}
\chi_{\opn{top}}(S)=\int_{S}\opn{td}(S)=\opn{RR}(S;0).
\end{align}
\end{conjecture}

\begin{theorem}[{de Medrano--Rinc\'on--Shaw'23}]
If $S$ is a cpt tropical 
surface admitting a
Delzant face structure, then
\begin{align}
\chi_{\opn{top}}(S)=\opn{RR}(S;D)=\frac{c_1(S)^2+c_2(S)}{12}.
\end{align}

\end{theorem}

\begin{question}
What is a geometrical meaning of $\opn{RR}(S;D)$ for 
a divisor $D$ on $S$ ?
\end{question}

\end{frame}

\begin{frame}

$X$: a complete (nonsingular) algebraic variety,

$D$: an (nonsingular) effective divisor on $X$.

The exact sequence
\begin{align}
0 \to \mathcal{O}_X(-D)\to
\mathcal{O}_X
\to \iota_{*}\mathcal{O}_D \to 0 \notag
\end{align}
gives 
\begin{align}
\chi (H^\bullet(X;\mathcal{O}_X(-D)))=
\chi (H^\bullet(X;\mathcal{O}_X))-
\chi (H^\bullet(D;\mathcal{O}_D))
\end{align}

We can easily pose the following conjecture:
\begin{conjecture}[Folklore?]
\label{conjecture-rr-euler}
Let $D$ be a tropical submanifold of codimension $1$ on
a cpt tropical manifold $X$. Then,
\begin{align}
\opn{RR}(X;-D)&=\chi_{c}(X\setminus D) \\
&=\chi(X)-\chi(D) \notag
\end{align}
where $\chi_{c}(X\setminus D)$
is the compactly supported Euler 
characteristic of $X\setminus D$. 
 
\end{conjecture}

\end{frame}

\begin{frame}

$X$: a tropical manifold,
$\opn{LC}_x X$: the local cone of $X$ at $x$.

\begin{definition}[{e.g. Gross--Shokrieh'21, dM--R--S'23}]

The (maximal) \emph{lineality space} 
$\opn{lin}(X,x)$ of $X$ at $x$ is
the maximal subspace $L$ of $T_x X$ s.t.
\begin{align}
L+\opn{LC}_x X =\opn{LC}_x X.
\end{align}

\end{definition}



\begin{definition}
A tropical submanifold $D$ of $X$ is in 
\emph{permissible position} if for any $x\in D$
\begin{align}
\opn{lineal}(D,x) \subsetneq
 \opn{lineal}(X,x).
\end{align}

\end{definition}




\end{frame}


\begin{frame}
\begin{example}
A point $p$ in a tropical curve $C$ is in permissible 
position iff $\opn{val}(p)=2$.
\end{example}
\end{frame}

\begin{frame}
\begin{conjecture}
\label{conjecture-rr-euler}
Let $D$ be a tropical submanifold of codimension $1$ on
a cpt tropical manifold $X$. 
If $D$ is in permissible position, then 
\begin{align}
\opn{RR}(X;D)&=\chi(X\setminus D).
\end{align}
\end{conjecture}

\begin{example}
If $X$ is a connected and cpt integral affine manifold,
then
\begin{align}
\chi (X\setminus D)&=(-1)^{\dim X}\chi_c(X\setminus D), \\
\opn{RR}(X;D)&=(-1)^{\dim X}\opn{RR}(X;-D).
\end{align}
In particular, 
\end{example}

\end{frame}

\begin{frame}
	
\end{frame}

\end{document}
